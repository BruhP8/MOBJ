\documentclass{article}

\usepackage[T1]{fontenc}
\usepackage[french]{babel}
\usepackage[margin=2cm, top=1.5cm, bottom=2cm]{geometry}

\newcommand \Term{\texttt{Term}}
\newcommand \Xml{\texttt{::fromXml()}}
\newcommand \Net{\texttt{Net}}
\newcommand \Node{\texttt{Node}}
\newcommand \Instance{\texttt{Instance}}

\author{DUPUY, LEGOUEIX}
\title{UE MOBJ : Compte-rendu TME6}

\begin{document}

\maketitle

\section{\Term \Xml}

Dans \Term\Xml\ on crée un \Term. \\
Objet simple : donc pas besoin de boucle \texttt{while(true)}.\\
On instancie les attributs les uns après les autres en fait appelle à la
la primitive \texttt{xmlTextReaderGetAttribute()}.

\section{\Instance \Xml}
Idem que pour \Term \Xml.

\section{\Net\Xml}
%L'objet \Net\ étant un objet composé, il est nécessaire de boucler
%afin de pouvoir récupérer tous les \Node\ de sa liste.\\
Nous pourrions écrire un roman sur les raisons qui nous ont ammené à trouver
une solution pour le cas du \Net\.\\
Mais nous dirons seulement que de vouloir faire la même boucle que dans
\texttt{Cell::fromXml()} était une mauvaise idée, et que la veille au soir
à très exactement 23h30 nous trouvâmes la solution à ce problème.\\
En effet, la solution était bien plus simple ! Il suffisait simplement de
reprendre la structure de la fonction \Xml\ comme définie dans \Term\ , 
\Instance\ ou bien \Node\ , puis d'ajouter une boucle après avoir créé le \Net\.\\
Maintenant que nous savons la vérité, notre regard sur notre survie dans
l'UE MOBJ a changé. Tout est possible !

\section{\Node\Xml}
La primitive de \Node\Xml\ est différente de celle des autres
objet de la \texttt{Netlist} car l'objet \Node\ lors de sa création
doit effectuer la liaison du \Net\ auquel il est rattaché avec le
\Term\ qui l'encapsule.\\
De plus, comme les \Term\ ont automatiquement créé leur \Node , il s'agit
alors de mettre à jour les attributs du \Node\ encapsulé.

\end{document}